%%=============================================================================
%% Voorwoord
%%=============================================================================

\chapter*{\IfLanguageName{dutch}{Woord vooraf}{Preface}}%
\label{ch:voorwoord}

%% Het voorwoord is het enige deel van de bachelorproef waar je vanuit je
%% eigen standpunt (``ik-vorm'') mag schrijven. Je kan hier bv. motiveren
%% waarom jij het onderwerp wil bespreken.
%% Vergeet ook niet te bedanken wie je geholpen/gesteund/... heeft
Als fanclubfotograaf en trouwe supporter van Lindemans Aalst is deze club al jarenlang een vaste waarde in mijn leven. Het spreekt dan ook voor zich dat ik met veel enthousiasme mijn bachelorproef wou wijden aan een onderwerp dat me nauw aan het hart ligt.

Voor de totstandkoming van deze bachelorproef heb ik dan ook op de steun van verschillende mensen mogen rekenen, aan wie ik graag mijn oprechte dank wil uitspreken. In het bijzonder wil ik mijn co-promotor, Dhr. Joost Van Kerckhove, bedanken voor zijn bereidwillige medewerking en de gedeelde expertise. Eveneens een welgemeende dank aan de staf van Lindemans Aalst voor het aanleveren van de nodige statistieken, gegevens en achtergrondinformatie die deze bachelorproef mee vorm hebben gegeven.

Daarnaast wil ik mevrouw Lotte Van Steenberghe bedanken voor haar begeleiding en ondersteuning tijdens het hele traject.

Ook een bijzonder woord van dank aan mijn ouders en zus. Hun onvoorwaardelijke steun, eindeloze aanmoediging en begripvolle luisterend oor zijn voor mij van onschatbare waarde geweest in deze periode. Ze hebben me de rust, motivatie en ruimte gegeven om dit project tot een goed einde te brengen.

Tot slot wil ik ook mijn vriend Lennert bedanken, die steeds klaarstond om mijn vele vragen over de statistieken te beantwoorden.

Aan iedereen die op een of andere manier heeft bijgedragen: dankjewel. Zonder jullie hulp was dit werk niet geworden wat het nu is.

\vspace{5mm} %5mm vertical space

Ik wens u veel leesplezier toe!

Youna Noynaert