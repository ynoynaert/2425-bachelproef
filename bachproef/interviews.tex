\chapter{\IfLanguageName{dutch}{Interviews en requirementsanalyse}{Interviews and requirement analysis}}%
\label{ch:interviews}

In dit hoofdstuk worden de resultaten besproken van de interviews met de technische staf en bestuur van Lindemans Aalst. Het doel is om de functionele en technische eisen in kaart te brengen die de club stelt aan een geautomatiseerd statistisch analysesysteem. De inzichten uit deze gesprekken vormen de basis voor een requirementsdocument, dat op zijn beurt dient als uitgangspunt voor de vergelijkende analyse van AI-systemen in de volgende fase.

De vragen zijn gestructureerd rond vier kernonderwerpen: functionele en technische eisen, gebruik van AI en automatisering, barrières en implementatie-uitdagingen en gebruiksvriendelijkheid en toegankelijkheid. Het is belangrijk te vermelden dat niet iedereen elke vraag kreeg. Per functie werden de juiste vragen gesteld, zodat de antwoorden specifiek aansluiten bij de rol en verantwoordelijkheid van de geïnterviewde.

Door deze gestructureerde aanpak wordt een helder beeld geschetst van de verwachtingen en behoeften van de technische staf, wat cruciaal is voor de verdere ontwikkeling en selectie van een geschikt AI-ondersteund statistisch systeem.

\section{Inzichten van stakeholders}

\subsection{Functionele en technische eisen}
\begin{enumerate}
  \item Welke statistieken zijn voor jullie het belangrijkst bij het analyseren van wedstrijden en trainingen?
  \subsubsection{Interview met de coach}
  Bij de analyse van wedstrijden en de scouting van tegenstanders worden verschillende prestatie-indicatoren geëvalueerd. Belangrijke factoren zijn de scoringspercentages in side-out en transitie, de gemaakte fouten en de rotaties waarin deze het meest voorkomen. Daarnaast wordt geanalyseerd in welke rotaties de tegenstander het meest effectief is en welke spelers in specifieke situaties het vaakst worden aangespeeld.

  De rol van de setter is hierbij essentieel. Er wordt onderzocht welke keuzes hij maakt in verschillende recepties en rotaties, en in welke posities de tegenstander het hoogste scoringspercentage behaalt. Verder wordt gekeken naar de richtingen van de aanvallers, de verdedigingspatronen en de servicevariaties.
  
  Bij de vergelijking met het eigen team ligt de focus vooral op de receptie. Hierbij wordt geanalyseerd hoe spelers de service verwerken (links, rechts, boven- of onderhands) en of er zwakke punten zijn, zoals moeite met korte of diepe ballen of een voorkeur voor een specifieke kant. Deze informatie helpt om strategieën te optimaliseren en het eigen spel te verbeteren.
  \subsubsection{Interview met de assistent-coach}
  Bij de analyse van een team worden zowel de scoringpercentages uit side-out als uit transitie (wanneer de rally bezig is) bekeken. Dit gebeurt zowel op groepsniveau als per rotatie, zodat we inzicht krijgen in welke rotaties moeilijker of makkelijker zijn voor het team. Het is van belang om te weten waar de sterke en zwakke plekken per rotatie liggen.

  Daarnaast wordt de individuele prestaties van spelers geanalyseerd, zowel in termen van wiskundige scores als in hun bijdrage aan het team. De verdeling van de setter wordt nauwkeurig geobserveerd, waarbij gekeken wordt of hij bepaalde spelers vaker of minder vaak aanspeelt. Het kan opvallen of een speler minder vaak wordt aangespeeld, wat mogelijk wijst op een specifieke tactiek of voorkeur van de setter.

  Bij de tegenstander wordt eenzelfde analyse uitgevoerd, maar hier ligt de nadruk minder op individuele percentages, tenzij een speler uitblinkt met slechte cijfers in receptie. Dit is belangrijk om te begrijpen of een bepaalde speler mogelijk meer ballen ontvangt, ondanks een zwakke receptie, en hoe dit de tegenstander beïnvloedt.

  Door deze gegevens kunnen we inschatten wie de grootste kans heeft om de pas te krijgen per rotatie en hoe de tegenstander zijn setter verdeelt. Dit helpt bij het bepalen van strategische aanpassingen en het identificeren van zwakke plekken in de verdediging van zowel ons eigen team als dat van de tegenstander.
  \item Welke data wordt momenteel handmatig verzameld, en welke processen zouden jullie willen automatiseren?
  \subsubsection{Interview met de coach}
  Scouting wordt steeds verder geautomatiseerd, waarbij data-invoer leidt tot automatische analyses, zoals de aanvalsrichtingen van spelers en de settercalls. Als de scouting goed wordt uitgevoerd, biedt dit direct inzicht in de verdedigingspatronen en aanvalsvoorkeuren van de tegenstander.

  Hoewel automatisering veel werk uit handen neemt, blijft videobeelden analyseren een belangrijk voordeel. Hiermee kunnen specifieke patronen, zoals de aanvalsrichting van een speler, duidelijk worden geïdentificeerd. Scoutingsoftware en analysesystemen zorgen ervoor dat veel werk al vooraf wordt gedaan, waardoor de rol van handmatige scouting steeds kleiner wordt.

  Als technologie zich verder ontwikkelt, zou scouting op termijn mogelijk volledig geautomatiseerd kunnen worden. Dit zou het proces efficiënter maken en de behoefte aan traditionele scouts verminderen.
  \subsubsection{Interview met de assistent-coach}
  De data wordt ingevoerd door de scouter en omvat gedetailleerde informatie over de receptie van de spelers, zoals of ze voor, achter, links of rechts van zich de bal ontvangen. Het proces begint al bij de opslag: waar de opslag vandaan komt en waar de bal naartoe gaat, gevolgd door een analyse van de receptie en waar de speler de bal heeft ontvangen. Deze gedetailleerde informatie biedt een helder beeld van hoe goed een speler presteert in verschillende situaties.

  Met deze gegevens kun je de serve aanpassen en gericht kiezen welk doelwit je probeert te bereiken, afhankelijk van de sterkte en zwaktes van de tegenstander. De scouter levert dus een cruciale hoeveelheid informatie, die, als deze automatisch zou kunnen worden verzameld, ideaal zou zijn.

  Als het mogelijk is om de basis van deze gegevens geautomatiseerd te ontvangen, zou dat al een enorme vooruitgang zijn. Dit maakt het mogelijk om snel met de gegevens aan de slag te gaan, zelfs tijdens een training, zonder dat er een scouter nodig is. Het geautomatiseerd verkrijgen van deze informatie zou het proces veel efficiënter maken en stelt het team in staat om sneller en effectiever te analyseren en aan te passen.
  \item Hoe gedetailleerd moeten de statistieken zijn? Zijn er specifieke metrics die momenteel ontbreken?
  \subsubsection{Interview met de coach}
  Hoewel scouting steeds meer wordt geautomatiseerd, zullen er altijd fouten blijven bestaan. De kwaliteit van de data hangt sterk af van de vaardigheden en ervaring van de scouter. Een goede scouter verzamelt nauwkeurigere en gedetailleerdere informatie dan iemand zonder expertise.

  Het gebruik van één programma kan het aantal fouten verminderen, maar het blijft noodzakelijk om de gegevens te controleren en te vergelijken met de invoer van de huidige scouter. Door deze vergelijking kan worden bepaald in hoeverre de data betrouwbaar is en waar eventuele correcties nodig zijn.
  \subsubsection{Interview met de assistent-coach}
  Zo gedetailleerd mogelijk. Als het minder is, zou dit weinig voordeel opleveren. Het doel is om de data zo volledig mogelijk te hebben, zodat er geen belangrijke informatie verloren gaat. Dit helpt bij het maken van betere analyses en strategische beslissingen.

  DataVolley geeft een goed overzicht van alle statistieken. Soms zijn er wat technische fouten bij DataVolleyb zodat er niet kan samen gewerkt worden met de scouter. Ook de kwaliteit van de internetverbinding speelt een rol voor de samenwerking.
  \item Welke outputformaten (dashboards, rapporten, live feedback) zijn het meest bruikbaar voor de technische staf?
  \subsubsection{Interview met de coach}
  Scoutinggegevens geven inzicht in verschillende spelaspecten, zoals aanvalsrichtingen en de verdeling van de setter onder verschillende omstandigheden (goede, minder goede en slechte situaties, evenals transitie). Deze informatie wordt niet alleen voorafgaand aan de wedstrijd geanalyseerd, maar ook tijdens de wedstrijd gebruikt om aanpassingen te maken.

  Tijdens de wedstrijd wordt onder andere gekeken naar servicepatronen en targets. Dit helpt bij het bepalen of er verschuivingen nodig zijn in de verdediging. Scouters zoals Joost en Erwin leveren deze informatie aan, samen met match- en setrapporten waarin wordt geanalyseerd welke patronen het meest zijn gespeeld. Deze gegevens vormen de basis voor verdere tactische beslissingen.
  \subsubsection{Interview met de assistent-coach}
  Het is belangrijk dat het snel en eenvoudig toegankelijk is. De leesbaarheid is ook belangrijk. Het is ook belangrijk dat de software op alle mogelijke computersystemen kan werken.
  \item Hoe vaak moeten de statistieken bijgewerkt worden? Is real-time verwerking gewenst?
  \subsubsection{Interview met de coach}
  Real-time scouting is vooral tijdens de wedstrijd gewenst, idealiter zelfs binnen een set. Dit kan waardevolle inzichten opleveren, maar door het hoge tempo van het spel is het niet altijd even bruikbaar. Snelle analyses kunnen helpen bij tactische aanpassingen, maar de beperkte tijd maakt het moeilijk om alles direct te verwerken en toe te passen.
  \subsubsection{Interview met de assistent-coach}
  Dit moet real-time zijn. Ten alle tijden moet de percentages van de spelers bekend zijn. Dit is belangrijk voor de coach en de spelers. De coach kan dan ook sneller ingrijpen als het nodig is.
\end{enumerate}

\subsection{Gebruik van AI en automatisering}
\begin{enumerate}
  \item Welke technologieën gebruiken jullie momenteel voor statistieken en videoanalyse?
  \subsubsection{Interview met de assistent-coach}
  Momenteel wordt DataVolley gebruikt voor de statistieken. Volgend seizoen zouden we ook VolleyMetrics willen gebruiken. Dit systeem zorgt ervoor dat spelers hun eigen data kunnen bekijken, maar ook die van de tegenstander van thuis uit. Dit is belangrijk voor de spelers, omdat ze dan ook zelf kunnen zien hoe ze presteren en waar ze op moeten letten tijdens de volgende match.
  \item Wat zijn de grootste voordelen en nadelen van de huidige systemen?
  \subsubsection{Interview met de assistent-coach}
  De nadelen zijn dat het niet automatisch is. Als de scouter niet aanwezig is, kan er geen data verzameld worden. De foutenlast van de andere scouter is niet geweten. De scouter geeft zelf de data in en bepaalt zelf hoe goed of slecht de speler speelt. De interpretatie is altijd anders. De data van de andere ploeg wordt eerst nog nagekeken op fouten, sommige worden niet gecorrigeerd. Dit kan leiden tot een verkeerde analyse van de tegenstander.
  \item In hoeverre staan de club open voor het gebruik van AI bij data-analyse?
  \subsubsection{Interview met de manager}
  De club staat zeker open voor nieuwe technologieën die kunnen helpen, zolang ze bijdragen aan betere prestaties. Een belangrijke factor blijft echter de kosten. De investering moet opwegen tegen de meerwaarde die het biedt. Het idee om zulke tools vanuit de liga overkoepelend verplicht te stellen, lijkt een goede benadering. Een testfase bij bepaalde clubs zou waardevolle inzichten opleveren, waarna het systeem breder geïmplementeerd kan worden, zodat alle teams er uiteindelijk van kunnen profiteren.
  \item Hebben jullie eerder ervaring met AI-gestuurde sportstatistieken? Zo ja, hoe waren die ervaringen?
  \subsubsection{Interview met de coach}
  Er is geen ervaring met AI-gestuurde sportstatistieken.
  \subsubsection{Interview met de assistent-coach}
  Er is geen ervaring met AI-gestuurde sportstatistieken.
  \subsubsection{Interview met de manager}
  Er is geen ervaring met AI-gestuurde sportstatistieken.
  \item Welke taken zouden jullie het liefst volledig geautomatiseerd zien?
  \subsubsection{Interview met de coach}
  Richtingen van aanvallers en de beslissingen van de setter tijdens wedstrijden is wel belangrijk. Daar kan echt beslissingen op gebasseerd worden. 
  \subsubsection{Interview met de assistent-coach}
  Het zou ideaal zijn als er bepaalde functies automatisch uitgevoerd kunnen worden. Momenteel staan die functies ook in DataVolley, maar die moeten handmatig uitgevoerd worden.
  \subsubsection{Interview met de manager}
  Het zou ideaal zijn als het scoutingproces volledig geautomatiseerd is, zodat gegevens direct en zonder handmatige invoer beschikbaar zijn. Dit zou niet alleen de efficiëntie verbeteren, maar ook personeelskosten besparen, aangezien minder tijd besteed hoeft te worden aan handmatige data-invoer en analyse. Een systeem dat automatisch de juiste informatie verzamelt en verwerkt, maakt het voor clubs gemakkelijker om snel en effectief te reageren zonder extra middelen te moeten inzetten.
\end{enumerate}


\subsection{Barrières en implementatie-uitdagingen}
\begin{enumerate}
  \item Wat zijn de grootste uitdagingen bij het implementeren van nieuwe technologieën in jullie workflow?
  \subsubsection{Interview met de coach}
  Er is openheid voor nieuwe scoutingmethodes, maar een solide basiskennis blijft essentieel. Een belangrijk vergelijkingspunt is hoe de nieuwe methode zich verhoudt tot bestaande systemen zoals DataVolley. Als een nieuw systeem al goed ontwikkeld is en vroeg kan worden geïmplementeerd, zullen er waarschijnlijk weinig problemen optreden. Dit maakt snelle aanpassingen mogelijk. Uiteindelijk moet het systeem een verbetering zijn ten opzichte van de huidige werkwijze om effectief ingezet te worden.
  \subsubsection{Interview met de assistent-coach}
  Er moet op het systeem vertrouwd kunnen worden. De data die nu verkregen wordt door de scouter moet iedentiek zijn aan die dat het systeem geeft. Ook de prijs van het syteem is een uitdaging. Als de AI perfect doet wat er wordt verwacht, maar de prijs is the hoog zal dit niet worden geïmplementeerd. Op training kan er dan gekeken worden naar een goedkopere oplossing, maar voor op een match is dit absoluut niet mogelijk.
  \subsubsection{Interview met de manager}
  Een belangrijk aandachtspunt bij het implementeren van geautomatiseerde systemen is de betrouwbaarheid, vooral tijdens wedstrijden. Aangezien de club veel met vrijwilligers werkt die niet altijd de technische kennis hebben om problemen op te lossen, kan een technisch probleem (zoals het uitvallen van het systeem) tijdens een match grote gevolgen hebben, omdat er dan geen statistieken meer beschikbaar zijn. Om dit te voorkomen, zou het handig zijn om een persoon met een technische achtergrond aan te stellen die in geval van problemen snel kan ingrijpen en de benodigde ondersteuning biedt. Dit garandeert dat het systeem soepel blijft draaien en de scoutingdata op cruciale momenten toegankelijk blijft.
  \item Is er weerstand binnen de club tegen het gebruik van AI voor statistieken? Waarom wel/niet?
  \subsubsection{Interview met de manager}
  Er is geen weerstand tegen het gebruik van AI voor statistieken, zolang het systeem betrouwbaar is en de kosten in verhouding staan tot de voordelen. De club staat open voor nieuwe technologieën die kunnen helpen bij het verbeteren van prestaties en scoutingprocessen.
  \item Hoeveel technische kennis is er binnen de staf aanwezig voor het werken met geavanceerde analysesystemen?
  \subsubsection{Interview met de coach}
  Er is een vrij grote technische kennis aanwezig bij de sportieve staf.
  \subsubsection{Interview met de assistent-coach}
  Er is een vrij grote technische kennis aanwezig bij de sportieve staf.
  \subsubsection{Interview met de manager}
  De club heeft een aantal mensen met technische kennis, maar het is belangrijk dat deze kennis ook beschikbaar is voor de vrijwilligers die met het systeem werken. Dit kan door training of ondersteuning te bieden aan de vrijwilligers, zodat zij ook in staat zijn om het systeem effectief te gebruiken en eventuele problemen op te lossen.

  Daarnaast zou het nuttig zijn om bij het opbouwen van de zaal altijd iemand met technische expertise aanwezig te hebben. Deze persoon kan ervoor zorgen dat alles correct wordt ingesteld en goed functioneert, zodat er geen technische problemen ontstaan die het verzamelen van statistieken tijdens de wedstrijd kunnen verstoren.
  \item Welke ondersteuning of training zou nodig zijn om het systeem optimaal te benutten?
  \subsubsection{Interview met de manager}
  Om ervoor te zorgen dat alle vrijwilligers effectief met het systeem kunnen werken, is het belangrijk om meerdere trainingssessies aan te bieden. Dit helpt niet alleen om de kennis te verspreiden, maar zorgt er ook voor dat iedereen vertrouwd raakt met het systeem en in staat is om eventuele technische problemen op te lossen. Het is duidelijk dat niet iedereen dezelfde technische kennis heeft, dus herhalende sessies kunnen ervoor zorgen dat alle vrijwilligers op hetzelfde niveau staan en beter voorbereid zijn om het systeem optimaal te gebruiken.
  \item Zijn er specifieke privacy- of ethische bezwaren binnen de club met betrekking tot geautomatiseerde dataverzameling?
  \subsubsection{Interview met de manager}
  Er zijn geen bezwaren tegen geautomatiseerde dataverzameling. De General Data Protection Regulation (GDPR) heeft hier eigenlijk geen betrekkingen op.
\end{enumerate}

\subsection{Gebruiksvriendelijkheid en toegankelijkheid}
\begin{enumerate}
  \item Wat zijn de belangrijkste eigenschappen die een nieuw systeem gebruiksvriendelijk maken?
  \item Hoe snel moet iemand zonder technische achtergrond met het systeem kunnen werken?
  \subsubsection{Interview met de assistent-coach}
  Het hangt af van de taak van de persoon. De coach kan terugvallen op de assistent, maar wordt erop afgerekend als hij het niet kent als het niet goed gaat. Als assistent-coach moet je er eigenlijk ook al mee kunnen werken omdat er binnen de seconde iets kunnen oproepen.
  \item Zijn er voorkeuren qua interface of visuele weergave van de data?
  \subsubsection{Interview met de coach}
  Elke vorm van informatie kan bijdragen aan betere analyses en tactische beslissingen. Hoe meer data beschikbaar is, hoe beter het team kan inspelen op verschillende spelsituaties. Alle extra inzichten, of het nu gaat om aanvalsrichtingen, servicepatronen of spelverdelingen, kunnen helpen bij het optimaliseren van de strategie. Daarom is elke aanvulling op de bestaande scouting welkom, mits deze betrouwbaar en praktisch toepasbaar is.
  \subsubsection{Interview met de assistent-coach}
  Het is belangrijk dat de interface gebruiksvriendelijk is en dat de data op een duidelijke manier wordt gepresenteerd. Dit maakt het gemakkelijker om snel de benodigde informatie te vinden en te begrijpen. Een overzichtelijke interface helpt ook om de gegevens effectief te analyseren en strategische beslissingen te nemen. Kleur codering kan ook nuttig zijn om snel belangrijke informatie te identificeren en te interpreteren.
  \item Moet het systeem geïntegreerd kunnen worden met andere software die jullie al gebruiken?
  \subsubsection{Interview met de coach}
  Het is toch belangrijk dat iedereen met hetzelfde systeem werkt om consistentie en efficiëntie te waarborgen. Toch is er ruimte voor aanvullende tools of losse systemen, zolang ze een meerwaarde bieden. Idealiter zouden nieuwe methodes geïntegreerd kunnen worden in DataVolley, zodat alle informatie centraal blijft. Samenwerking tussen verschillende scoutingtools en systemen zou alleen maar voordelen opleveren, mits de integratie soepel verloopt en de bruikbaarheid gewaarborgd blijft.
  \subsubsection{Interview met de assistent-coach}
  Dan is er nog de combinatie en kan er gelijdelijk aan aanpassen naar een AI gestuurd systeem. Dat dan echt het vertrouwen heeft van iedereen. Voor een training zou het wel gemakkelijker zijn dat een apart systeem is. Dan is er geen nood aan de data die tijdens een match wordt geanalyseerd.
  \subsubsection{Interview met de manager}
  Het delen van statistieken tussen clubs is verplicht voor een transparante en gezamenlijke benadering van scouting en wedstrijdanalyse. Daarom zou het logisch zijn dat het nieuwe systeem goed integreerbaar is met andere software, zodat gegevens naadloos gedeeld kunnen worden. Dit zorgt ervoor dat de informatie consistent blijft en dat alle betrokken clubs eenvoudig toegang hebben tot dezelfde data, wat de samenwerking en het gebruik van de scoutingtools aanzienlijk vergemakkelijkt.
  \item Hoe belangrijk is mobiele toegankelijkheid (bijv. via tablets of smartphones)?
  \subsubsection{Interview met de coach}
  Aangezien DataVolley veel wordt gebruikt, blijft het een belangrijke standaard binnen de scouting. Echter, de beperkte bruikbaarheid op tablets maakt het minder toegankelijk in sommige situaties. Een gebruiksvriendelijker en compacter systeem zou een grote verbetering zijn, vooral als het beter aansluit bij de moderne technologieën. De vooruitgang in scoutingsoftware laat zien dat er steeds efficiëntere en kleinere oplossingen mogelijk zijn, wat de praktische toepasbaarheid ten goede komt.
  \subsubsection{Interview met de assistent-coach}
  Het is belangrijk dat het systeem ook op tablets of smartphones toegankelijk is. Dit maakt het gemakkelijker om gegevens te bekijken en te analyseren. Mobiele toegankelijkheid vergroot de flexibiliteit en zorgt ervoor dat je altijd en overal toegang hebt tot belangrijke informatie.
\end{enumerate}

\chapter{\IfLanguageName{dutch}{Interviews en requirementsanalyse}{Interviews and requirement analysis}}%
\label{ch:requirementsanalyse}

In dit hoofdstuk worden de resultaten besproken van de interviews met de technische staf en bestuur van Lindemans Aalst. Het doel is om de functionele en technische eisen in kaart te brengen die de club stelt aan een geautomatiseerd statistisch analysesysteem. De inzichten uit deze gesprekken vormen de basis voor een requirementsdocument, dat op zijn beurt dient als uitgangspunt voor de vergelijkende analyse van AI-systemen in de volgende fase.

De vragen zijn gestructureerd rond vier kernonderwerpen: functionele en technische eisen, gebruik van AI en automatisering, barrières en implementatie-uitdagingen en gebruiksvriendelijkheid en toegankelijkheid. Het is belangrijk te vermelden dat niet iedereen elke vraag kreeg. Per functie werden de juiste vragen gesteld, zodat de antwoorden specifiek aansluiten bij de rol en verantwoordelijkheid van de geïnterviewde.

Door deze gestructureerde aanpak wordt een helder beeld geschetst van de verwachtingen en behoeften van de technische staf, wat cruciaal is voor de verdere ontwikkeling en selectie van een geschikt AI-ondersteund statistisch systeem.

\section{Inzichten van stakeholders}

\subsection{Functionele en technische eisen}
\begin{enumerate}
  \item Welke statistieken zijn voor jullie het belangrijkst bij het analyseren van wedstrijden en trainingen?
  \subsubsection{Interview met de coach}
  Bij de analyse van wedstrijden en de scouting van tegenstanders worden verschillende prestatie-indicatoren geëvalueerd. Belangrijke factoren zijn de scoringspercentages in side-out en transitie, de gemaakte fouten en de rotaties waarin deze het meest voorkomen. Daarnaast wordt geanalyseerd in welke rotaties de tegenstander het meest effectief is en welke spelers in specifieke situaties het vaakst worden aangespeeld.

  De rol van de setter is hierbij essentieel. Er wordt onderzocht welke keuzes hij maakt in verschillende recepties en rotaties, en in welke posities de tegenstander het hoogste scoringspercentage behaalt. Verder wordt gekeken naar de richtingen van de aanvallers, de verdedigingspatronen en de servicevariaties.
  
  Bij de vergelijking met het eigen team ligt de focus vooral op de receptie. Hierbij wordt geanalyseerd hoe spelers de service verwerken (links, rechts, boven- of onderhands) en of er zwakke punten zijn, zoals moeite met korte of diepe ballen of een voorkeur voor een specifieke kant. Deze informatie helpt om strategieën te optimaliseren en het eigen spel te verbeteren.

  \item Welke data wordt momenteel handmatig verzameld, en welke processen zouden jullie willen automatiseren?
  \subsubsection{Interview met de coach}
  Scouting wordt steeds verder geautomatiseerd, waarbij data-invoer leidt tot automatische analyses, zoals de aanvalsrichtingen van spelers en de settercalls. Als de scouting goed wordt uitgevoerd, biedt dit direct inzicht in de verdedigingspatronen en aanvalsvoorkeuren van de tegenstander.

  Hoewel automatisering veel werk uit handen neemt, blijft videobeelden analyseren een belangrijk voordeel. Hiermee kunnen specifieke patronen, zoals de aanvalsrichting van een speler, duidelijk worden geïdentificeerd. Scoutingsoftware en analysesystemen zorgen ervoor dat veel werk al vooraf wordt gedaan, waardoor de rol van handmatige scouting steeds kleiner wordt.

  Als technologie zich verder ontwikkelt, zou scouting op termijn mogelijk volledig geautomatiseerd kunnen worden. Dit zou het proces efficiënter maken en de behoefte aan traditionele scouts verminderen.
  \item Hoe gedetailleerd moeten de statistieken zijn? Zijn er specifieke metrics die momenteel ontbreken?
  \subsubsection{Interview met de coach}
  Hoewel scouting steeds meer wordt geautomatiseerd, zullen er altijd fouten blijven bestaan. De kwaliteit van de data hangt sterk af van de vaardigheden en ervaring van de scouter. Een goede scouter verzamelt nauwkeurigere en gedetailleerdere informatie dan iemand zonder expertise.

  Het gebruik van één programma kan het aantal fouten verminderen, maar het blijft noodzakelijk om de gegevens te controleren en te vergelijken met de invoer van de huidige scouter. Door deze vergelijking kan worden bepaald in hoeverre de data betrouwbaar is en waar eventuele correcties nodig zijn.
  \item Welke outputformaten (dashboards, rapporten, live feedback) zijn het meest bruikbaar voor de technische staf?
  \subsubsection{Interview met de coach}
  Scoutinggegevens geven inzicht in verschillende spelaspecten, zoals aanvalsrichtingen en de verdeling van de setter onder verschillende omstandigheden (goede, minder goede en slechte situaties, evenals transitie). Deze informatie wordt niet alleen voorafgaand aan de wedstrijd geanalyseerd, maar ook tijdens de wedstrijd gebruikt om aanpassingen te maken.

  Tijdens de wedstrijd wordt onder andere gekeken naar servicepatronen en targets. Dit helpt bij het bepalen of er verschuivingen nodig zijn in de verdediging. Scouters zoals Joost en Erwin leveren deze informatie aan, samen met match- en setrapporten waarin wordt geanalyseerd welke patronen het meest zijn gespeeld. Deze gegevens vormen de basis voor verdere tactische beslissingen.
  \item Hoe vaak moeten de statistieken bijgewerkt worden? Is real-time verwerking gewenst?
  \subsubsection{Interview met de coach}
  Real-time scouting is vooral tijdens de wedstrijd gewenst, idealiter zelfs binnen een set. Dit kan waardevolle inzichten opleveren, maar door het hoge tempo van het spel is het niet altijd even bruikbaar. Snelle analyses kunnen helpen bij tactische aanpassingen, maar de beperkte tijd maakt het moeilijk om alles direct te verwerken en toe te passen.
\end{enumerate}

\subsection{Gebruik van AI en automatisering}
\begin{enumerate}
  \item Welke technologieën gebruiken jullie momenteel voor statistieken en videoanalyse?
  \item Wat zijn de grootste voordelen en nadelen van de huidige systemen?
  \item In hoeverre staan de club open voor het gebruik van AI bij data-analyse?
  \subsubsection{Interview met de manager}
  De club staat zeker open voor nieuwe technologieën die kunnen helpen, zolang ze bijdragen aan betere prestaties. Een belangrijke factor blijft echter de kosten. De investering moet opwegen tegen de meerwaarde die het biedt. Het idee om zulke tools vanuit de liga overkoepelend verplicht te stellen, lijkt een goede benadering. Een testfase bij bepaalde clubs zou waardevolle inzichten opleveren, waarna het systeem breder geïmplementeerd kan worden, zodat alle teams er uiteindelijk van kunnen profiteren.
  \item Hebben jullie eerder ervaring met AI-gestuurde sportstatistieken? Zo ja, hoe waren die ervaringen?
  \subsubsection{Interview met de coach}
  Er is geen ervaring met AI-gestuurde sportstatistieken.
  \subsubsection{Interview met de manager}
  Er is geen ervaring met AI-gestuurde sportstatistieken.
  \item Welke taken zouden jullie het liefst volledig geautomatiseerd zien?
  \subsubsection{Interview met de coach}
  Richtingen van aanvallers en de beslissingen van de setter tijdens wedstrijden is wel belangrijk. Daar kan echt beslissingen op gebasseerd worden. 
  \subsubsection{Interview met de manager}
  Het zou ideaal zijn als het scoutingproces volledig geautomatiseerd is, zodat gegevens direct en zonder handmatige invoer beschikbaar zijn. Dit zou niet alleen de efficiëntie verbeteren, maar ook personeelskosten besparen, aangezien minder tijd besteed hoeft te worden aan handmatige data-invoer en analyse. Een systeem dat automatisch de juiste informatie verzamelt en verwerkt, maakt het voor clubs gemakkelijker om snel en effectief te reageren zonder extra middelen te moeten inzetten.
\end{enumerate}


\subsection{Barrières en implementatie-uitdagingen}
\begin{enumerate}
  \item Wat zijn de grootste uitdagingen bij het implementeren van nieuwe technologieën in jullie workflow?
  \subsubsection{Interview met de coach}
  Er is openheid voor nieuwe scoutingmethodes, maar een solide basiskennis blijft essentieel. Een belangrijk vergelijkingspunt is hoe de nieuwe methode zich verhoudt tot bestaande systemen zoals DataVolley. Als een nieuw systeem al goed ontwikkeld is en vroeg kan worden geïmplementeerd, zullen er waarschijnlijk weinig problemen optreden. Dit maakt snelle aanpassingen mogelijk. Uiteindelijk moet het systeem een verbetering zijn ten opzichte van de huidige werkwijze om effectief ingezet te worden.
  \subsubsection{Interview met de manager}
  Een belangrijk aandachtspunt bij het implementeren van geautomatiseerde systemen is de betrouwbaarheid, vooral tijdens wedstrijden. Aangezien de club veel met vrijwilligers werkt die niet altijd de technische kennis hebben om problemen op te lossen, kan een technisch probleem (zoals het uitvallen van het systeem) tijdens een match grote gevolgen hebben, omdat er dan geen statistieken meer beschikbaar zijn. Om dit te voorkomen, zou het handig zijn om een persoon met een technische achtergrond aan te stellen die in geval van problemen snel kan ingrijpen en de benodigde ondersteuning biedt. Dit garandeert dat het systeem soepel blijft draaien en de scoutingdata op cruciale momenten toegankelijk blijft.
  \item Is er weerstand binnen de club tegen het gebruik van AI voor statistieken? Waarom wel/niet?
  \subsubsection{Interview met de manager}
  Er is geen weerstand tegen het gebruik van AI voor statistieken, zolang het systeem betrouwbaar is en de kosten in verhouding staan tot de voordelen. De club staat open voor nieuwe technologieën die kunnen helpen bij het verbeteren van prestaties en scoutingprocessen.
  \item Hoeveel technische kennis is er binnen de staf aanwezig voor het werken met geavanceerde analysesystemen?
  \subsubsection{Interview met de coach}
  Er is een vrij grote technische kennis aanwezig bij de sportieve staf.
  \subsubsection{Interview met de manager}
  De club heeft een aantal mensen met technische kennis, maar het is belangrijk dat deze kennis ook beschikbaar is voor de vrijwilligers die met het systeem werken. Dit kan door training of ondersteuning te bieden aan de vrijwilligers, zodat zij ook in staat zijn om het systeem effectief te gebruiken en eventuele problemen op te lossen.

  Daarnaast zou het nuttig zijn om bij het opbouwen van de zaal altijd iemand met technische expertise aanwezig te hebben. Deze persoon kan ervoor zorgen dat alles correct wordt ingesteld en goed functioneert, zodat er geen technische problemen ontstaan die het verzamelen van statistieken tijdens de wedstrijd kunnen verstoren.
  \item Welke ondersteuning of training zou nodig zijn om het systeem optimaal te benutten?
  \subsubsection{Interview met de manager}
  Om ervoor te zorgen dat alle vrijwilligers effectief met het systeem kunnen werken, is het belangrijk om meerdere trainingssessies aan te bieden. Dit helpt niet alleen om de kennis te verspreiden, maar zorgt er ook voor dat iedereen vertrouwd raakt met het systeem en in staat is om eventuele technische problemen op te lossen. Het is duidelijk dat niet iedereen dezelfde technische kennis heeft, dus herhalende sessies kunnen ervoor zorgen dat alle vrijwilligers op hetzelfde niveau staan en beter voorbereid zijn om het systeem optimaal te gebruiken.
  \item Zijn er specifieke privacy- of ethische bezwaren binnen de club met betrekking tot geautomatiseerde dataverzameling?
  \subsubsection{Interview met de manager}
  Er zijn geen bezwaren tegen geautomatiseerde dataverzameling. De General Data Protection Regulation (GDPR) heeft hier eigenlijk geen betrekkingen op.
\end{enumerate}

\subsection{Gebruiksvriendelijkheid en toegankelijkheid}
\begin{enumerate}
  \item Wat zijn de belangrijkste eigenschappen die een nieuw systeem gebruiksvriendelijk maken?
  \item Hoe snel moet iemand zonder technische achtergrond met het systeem kunnen werken?
  \item Zijn er voorkeuren qua interface of visuele weergave van de data?
  \subsubsection{Interview met de coach}
  Elke vorm van informatie kan bijdragen aan betere analyses en tactische beslissingen. Hoe meer data beschikbaar is, hoe beter het team kan inspelen op verschillende spelsituaties. Alle extra inzichten, of het nu gaat om aanvalsrichtingen, servicepatronen of spelverdelingen, kunnen helpen bij het optimaliseren van de strategie. Daarom is elke aanvulling op de bestaande scouting welkom, mits deze betrouwbaar en praktisch toepasbaar is.
  \item Moet het systeem geïntegreerd kunnen worden met andere software die jullie al gebruiken?
  \subsubsection{Interview met de coach}
  Het is toch belangrijk dat iedereen met hetzelfde systeem werkt om consistentie en efficiëntie te waarborgen. Toch is er ruimte voor aanvullende tools of losse systemen, zolang ze een meerwaarde bieden. Idealiter zouden nieuwe methodes geïntegreerd kunnen worden in DataVolley, zodat alle informatie centraal blijft. Samenwerking tussen verschillende scoutingtools en systemen zou alleen maar voordelen opleveren, mits de integratie soepel verloopt en de bruikbaarheid gewaarborgd blijft.
  \subsubsection{Interview met de manager}
  Het delen van statistieken tussen clubs is verplicht voor een transparante en gezamenlijke benadering van scouting en wedstrijdanalyse. Daarom zou het logisch zijn dat het nieuwe systeem goed integreerbaar is met andere software, zodat gegevens naadloos gedeeld kunnen worden. Dit zorgt ervoor dat de informatie consistent blijft en dat alle betrokken clubs eenvoudig toegang hebben tot dezelfde data, wat de samenwerking en het gebruik van de scoutingtools aanzienlijk vergemakkelijkt.
  \item Hoe belangrijk is mobiele toegankelijkheid (bijv. via tablets of smartphones)?
  \subsubsection{Interview met de coach}
  Aangezien DataVolley veel wordt gebruikt, blijft het een belangrijke standaard binnen de scouting. Echter, de beperkte bruikbaarheid op tablets maakt het minder toegankelijk in sommige situaties. Een gebruiksvriendelijker en compacter systeem zou een grote verbetering zijn, vooral als het beter aansluit bij de moderne technologieën. De vooruitgang in scoutingsoftware laat zien dat er steeds efficiëntere en kleinere oplossingen mogelijk zijn, wat de praktische toepasbaarheid ten goede komt.
\end{enumerate}

