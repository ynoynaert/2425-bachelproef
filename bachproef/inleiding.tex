%%=============================================================================
%% Inleiding
%%=============================================================================

\chapter{\IfLanguageName{dutch}{Inleiding}{Introduction}}%
\label{ch:inleiding}

Datagedreven besluitvorming wordt steeds belangrijker in de sportwereld. Professionele sportteams maken gebruik van geavanceerde analysetools om prestaties te meten, tactieken te optimaliseren en spelersontwikkeling te bevorderen. In de volleybalwereld worden statistieken al decennialang gebruikt om inzicht te krijgen in spelpatronen en individuele prestaties. Traditioneel worden deze gegevens echter handmatig verzameld en geanalyseerd, wat tijdrovend is en ruimte laat voor menselijke fouten.

\section{\IfLanguageName{dutch}{Probleemstelling}{Problem Statement}}%
\label{sec:probleemstelling}

Hoewel veel sporten de transitie naar geautomatiseerde data-analyse al hebben gemaakt, gebeurt dit in het Belgische volleybal nog nauwelijks. Geen enkele club in Liga A maakt momenteel gebruik van AI-technologie om statistieken te verzamelen en analyseren. Hierdoor blijft een waardevolle bron van informatie grotendeels onbenut. Het ontbreken van geautomatiseerde statistieken zorgt ervoor dat beslissingen vaak gebaseerd worden op subjectieve observaties in plaats van op objectieve data. Dit kan leiden tot minder nauwkeurige analyses en gemiste optimalisatiemogelijkheden.

\section{\IfLanguageName{dutch}{Onderzoeksvraag}{Research question}}%
\label{sec:onderzoeksvraag}

Deze studie tracht een bestaande AI-oplossing te identificeren die de meeste voordelen biedt voor volleybalclub Lindemans Aalst, zowel tijdens wedstrijden als trainingen. De centrale onderzoeksvraag luidt als volgt: \textit{Welke bestaande AI-oplossing voor volleybalstatistieken biedt de meeste voordelen voor Lindemans Aalst, zowel tijdens wedstrijden als trainingen?}

Hiervoor werden de volgende vragen ook onderzocht:
\begin{itemize}
    \item Waarom zijn statistieken belangrijk in de sportwereld en specifiek in volleybal?
    \item Welke technologieën worden momenteel gebruikt voor het automatiseren van sportstatistieken?
    \item Hoe presteren bestaande AI-systemen voor het verzamelen en analyseren van volleybalstatistieken?
\end{itemize}

\section{\IfLanguageName{dutch}{Onderzoeksdoelstelling}{Research objective}}%
\label{sec:onderzoeksdoelstelling}

Het uiteindelijke doel is om te bepalen welke bestaande AI-oplossing het meest geschikt is voor de automatisering van volleybalstatistieken bij Lindemans Aalst. Dit gebeurt aan de hand van een vergelijkende studie van bestaande technologieën en een proof-of-concept binnen de clubcontext.

Het onderzoek richt zich op het identificeren van een oplossing die voldoet aan de volgende succescriteria:
\begin{itemize}
  \item \textbf{Nauwkeurigheid}: De AI-technologie moet minstens even accuraat zijn als handmatige registratie en idealiter een verbeterde databetrouwbaarheid bieden.
  \item \textbf{Snelheid}: De verwerkingstijd van statistieken moet aanzienlijk korter zijn dan bij manuele methoden, zodat coaches tijdens wedstrijden en trainingen in realtime bruikbare inzichten krijgen.
  \item \textbf{Gebruikersvriendelijkheid}: De technologie moet eenvoudig te gebruiken zijn voor coaches en data-analisten, zonder uitgebreide technische kennis.
  \item \textbf{Kostenefficiëntie}: De implementatie en het onderhoud van de technologie moeten financieel haalbaar zijn binnen de middelen van de club.
  \item \textbf{Toepasbaarheid}: De technologie moet compatibel zijn met de infrastructuur en werkwijze van Lindemans Aalst en gemakkelijk te integreren in hun dagelijkse werking.
\end{itemize}

Het beoogde eindresultaat van dit onderzoek is een verslag met aanbevelingen over de meest geschikte AI-oplossing, inclusief een analyse van de voor- en nadelen van de onderzochte systemen. Daarnaast wordt een proof-of-concept van de gekozen technologie uitgevoerd, waarbij het systeem wordt getest in een reële trainings- en wedstrijdomgeving. Op basis van deze praktijktest wordt een evaluatie gemaakt van de effectiviteit en de haalbaarheid van de implementatie op lange termijn.

Naast de directe voordelen voor Lindemans Aalst kan dit onderzoek ook als referentie dienen voor andere volleybalclubs in Liga A die overwegen om AI-technologie te integreren in hun statistiekenregistratie en wedstrijdanalyse.

\section{\IfLanguageName{dutch}{Opzet van deze bachelorproef}{Structure of this bachelor thesis}}%
\label{sec:opzet-bachelorproef}

% Het is gebruikelijk aan het einde van de inleiding een overzicht te
% geven van de opbouw van de rest van de tekst. Deze sectie bevat al een aanzet
% die je kan aanvullen/aanpassen in functie van je eigen tekst.

De rest van deze bachelorproef is als volgt opgebouwd:

In Hoofdstuk~\ref{ch:stand-van-zaken} wordt een overzicht gegeven van de stand van zaken binnen het onderzoeksdomein, op basis van een literatuurstudie.

In Hoofdstuk~\ref{ch:methodologie} wordt de methodologie toegelicht en worden de gebruikte onderzoekstechnieken besproken om een antwoord te kunnen formuleren op de onderzoeksvragen.

% Vul hier aan voor je eigen hoofstukken, één of twee zinnen per hoofdstuk
In Hoofdstuk~\ref{ch:interviews} wordt de requirementsanalyse besproken, waarbij de functionele en technische eisen van de AI-oplossing in kaart worden gebracht.

In Hoofdstuk~\ref{ch:vergelijkendestudie} wordt een vergelijkende studie uitgevoerd van de geselecteerde AI-systemen, waarbij hun prestaties worden geëvalueerd op basis van nauwkeurigheid, snelheid en gebruiksgemak.	

In Hoofdstuk~\ref{ch:poc} wordt de proof-of-concept besproken, waarin het aanbevolen systeem wordt geïmplementeerd en getest binnen de dagelijkse werking van Lindemans Aalst.

In Hoofdstuk~\ref{ch:conclusie}, tenslotte, wordt de conclusie gegeven en een antwoord geformuleerd op de onderzoeksvragen. Daarbij wordt ook een aanzet gegeven voor toekomstig onderzoek binnen dit domein.