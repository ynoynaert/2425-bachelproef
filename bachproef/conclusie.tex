%%=============================================================================
%% Conclusie
%%=============================================================================

\chapter{Conclusie}%
\label{ch:conclusie}

% Trek een duidelijke conclusie, in de vorm van een antwoord op de
% onderzoeksvra(a)g(en). Wat was jouw bijdrage aan het onderzoeksdomein en
% hoe biedt dit meerwaarde aan het vakgebied/doelgroep? 
% Reflecteer kritisch over het resultaat. In Engelse teksten wordt deze sectie
% ``Discussion'' genoemd. Had je deze uitkomst verwacht? Zijn er zaken die nog
% niet duidelijk zijn?
% Heeft het onderzoek geleid tot nieuwe vragen die uitnodigen tot verder 
%onderzoek?

Deze bachelorproef onderzocht welke bestaande AI-oplossing voor het automatiseren van volleybalstatistieken het meest geschikt is voor volleybalclub Lindemans Aalst. De aanleiding hiervoor was het ontbreken van geautomatiseerde dataverzameling binnen de club, wat de nauwkeurigheid, snelheid en bruikbaarheid van statistieken beperkte tijdens zowel trainingen als wedstrijden.

Uit de literatuurstudie bleek duidelijk dat AI en machine learning-technologieën al succesvol worden toegepast in verschillende sportdisciplines en aanzienlijke meerwaarde bieden in termen van prestatieanalyse, trainingsoptimalisatie en blessurepreventie. Er werd ook vastgesteld dat in de Liga A van het Belgische volleybal geen enkele club op dit moment gebruikmaakt van AI voor statistiekenverzameling, wat een strategisch voordeel biedt voor Lindemans Aalst om als pionier op te treden.

De requirementsanalyse, gevoed door interviews met coaches, scouters en stafleden, bracht de functionele en technische noden van de club in kaart. Hieruit bleek onder meer de nood aan real-time feedback, een hoge nauwkeurigheid van dataregistratie, gebruiksvriendelijkheid en integratiemogelijkheden met bestaande systemen zoals DataVolley. Daarnaast werd de wens uitgesproken om repetitieve taken te automatiseren en subjectieve fouten bij manuele invoer te minimaliseren.

De vergelijkende studie van bestaande AI-systemen identificeerde Balltime AI als het meest geschikte systeem voor de clubcontext, op basis van criteria zoals nauwkeurigheid, real-time verwerking, compatibiliteit, outputformaten en kosten. Dit systeem werd vervolgens getest in een proof-of-concept tijdens officiële wedstrijden.

Uit de analyse van de praktijktesten bleek dat Balltime AI in staat was om de meeste statistieken even accuraat en in sommige gevallen zelfs accurater dan de manuele methode te registreren. De verwerking gebeurde bovendien sneller en leverde realtime inzichten op die tijdens de wedstrijd benut konden worden. De feedback van de coaches bevestigde de gebruiksvriendelijkheid van het systeem en onderstreepte de meerwaarde van de automatische dataverwerking voor strategische besluitvorming.

Op basis van de resultaten van het onderzoek kan geconcludeerd worden dat automatisering van statistieken via AI een duidelijke meerwaarde biedt voor Lindemans Aalst. Niet alleen worden statistieken sneller en betrouwbaarder verzameld, maar het stelt de technische staf ook in staat om sneller te anticiperen op spelontwikkelingen. De club kan hiermee een voortrekkersrol opnemen binnen de Belgische volleybalwereld.

Tot slot biedt dit onderzoek ook aanbevelingen voor de verdere implementatie van AI binnen sportcontexten, met aandacht voor trainingsmogelijkheden, technische ondersteuning en integratie met bestaande software-infrastructuren. Verder onderzoek kan zich richten op het uitbreiden van AI-analyse naar individuele spelersontwikkeling en blessurepreventie, alsook de toepassing van AI in andere sportdisciplines.
