\chapter{\IfLanguageName{dutch}{Proof of concept en praktijktest}{Proof of concept and field tests}}
\label{ch:poc}
Na de literatuurstudie, requirementsanalyse en vergelijkende evaluatie van bestaande AI-systemen, vormt dit hoofdstuk een belangrijke schakel in het beantwoorden van de centrale onderzoeksvraag. Hierin wordt het geselecteerde AI-systeem geïmplementeerd in een reële context bij volleybalclub Lindemans Aalst als een proof of concept (PoC).

Deze fase heeft tot doel om de theoretische meerwaarde van het systeem in de praktijk te valideren. Tijdens trainingen en wedstrijden wordt geëvalueerd of het systeem voldoet aan de functionele en technische vereisten zoals geformuleerd in de voorgaande fases. Hierbij wordt bijzondere aandacht besteed aan de gebruiksvriendelijkheid, de nauwkeurigheid van dataverzameling, de snelheid van verwerking en de integratie in de bestaande werkwijze van de club.

Het oorspronkelijke plan voorzag een praktijktestperiode van twee weken. Echter, aangezien het volleybalseizoen onverwacht tot een einde kwam tijdens het midden van deze fase, kon de proof of concept slechts tijdens twee resterende wedstrijden worden uitgevoerd. Ondanks deze beperkte testperiode bood dit toch waardevolle inzichten in de toepasbaarheid en prestaties van het AI-systeem in een wedstrijdcontext.

Door de prestaties van het geautomatiseerde systeem te vergelijken met handmatig geregistreerde gegevens en door feedback van coaches en staf te verzamelen, wordt nagegaan of het AI-systeem daadwerkelijk een significante meerwaarde kan bieden voor de werking van de club. De inzichten uit deze praktijktest vormen een essentiële basis voor het formuleren van aanbevelingen omtrent een bredere implementatie op lange termijn.

In bijlage \ref{ch:afkortingen} is een overzicht te vinden van de afkortingen die in dit hoofdstuk worden gebruikt. De andere statistieken en vergelijkingen zijn te vinden in bijlage \ref{ch:statistieken}.

\section{Kwartfinale Play-offs - 16/4/2025}
\subsection{Vergelijking van de statistieken}
\subsubsection{Set 2 - Lindemans Aalst}
\label{sec:PL1_Aalst2}
% TODO: vergelijking
% TODO: andere stats ook bijzetten
\begin{figure}
  \centering
  \includegraphics[width=\textwidth]{PL1_AM/SET2/PL1_AM_2.png}
  \caption{\label{fig:PL1_AM_2}Resultaten van de manuele invoer van Lindemans Aalst in set 2.}
\end{figure}

\begin{table}[ht!]
  \centering
  \scriptsize
  \begin{tabular}{|l|c|c|c|c|c|c|c|c|c|c|c|c|} \hline
    \textbf{Name} & SA & SE & TA & Pct & Eff & Rtg & 0 & 1 & 2 & 3 & ? \\ \hline
    Timo Lohmus & 1 & 1 & 4 & 0.75 & 0 & 1.75 & 1 & 2 & 1 & 0 & 1 \\
    Max Schulz & 0 & 1 & 3 & 0.67 & -0.33 & 2.33 &  & 1 &  & 2 & 0 \\
    Hiago Crins & 0 & 0 & 3 & 1 & 0 & 1.67 &  & 2 &  & 1 & 0 \\
    Mihkel Varblane & 0 & 0 & 2 & 1 & 0 & 1 &  & 1 &  &  & 0 \\
    Alvaro Gimeno Rubio & 1 & 1 & 7 & 0.86 & 0 & 1.6 & 1 & 2 &  & 2 & 0 \\
    Robbe Ponseele & 1 & 1 & 2 & 0.5 & 0 & 1.5 & 1 &  &  & 1 & 0 \\
    Lucas Lorente López & 0 & 0 & 3 & 1 & 0 & 2.33 &  & 1 &  & 2 & 0 \\
    Bert Dufraing &  &  &  &  &  &  & 1 &  & 1 &  & 2 \\
    Lindemans Aalst & 0 & 5 & 22 & 0.77 & -0.23 & 2.18 &  & 6 & 6 & 10 & 0 \\
    Greenyard Maaseik & 3 & 4 & 24 & 0.83 & -0.04 & 1.81 & 3 & 7 & 2 & 9 & 0 \\ \hline
  \end{tabular}
  \caption[Serve statistieken gemaakt door Balltime AI voor Lindemans Aalst in set 2]{\label{tab:PL1ServeAalst2}Serve statistieken gemaakt door Balltime AI voor Lindemans Aalst in set 2.}
\end{table}

\begin{table}[ht!]
  \centering
  \scriptsize
  \begin{tabular}{|l|c|c|c|c|c|c|c|c|c|} \hline
    \textbf{Name} & 3 & 2 & 1 & 0 & TA & ? & Pass\% & Perfect PP\% & Good GP\% \\ \hline
    Timo Lohmus & 4 & 3 &  &  & 8 & 0 & 1.75 & 0.12 & 0.62 \\
    Max Schulz & 2 & 2 &  &  & 7 & 0 & 2.14 & 0.43 & 0.71 \\
    Hiago Crins &  &  &  &  &  &  &  &  &  \\
    Mihkel Varblane &  &  &  &  &  &  &  &  &  \\
    Alvaro Gimeno Rubio &  &  &  &  &  &  &  &  &  \\
    Robbe Ponseele &  &  &  &  &  &  &  &  &  \\
    Lucas Lorente López &  &  &  &  &  &  &  &  &  \\
    Bert Dufraing &  &  &  &  &  &  &  &  &  \\
    Lindemans Aalst & 5 & 2 & 7 &  & 17 & 3 & 1.86 & 0.36 & 0.5 \\
    Greenyard Maaseik & 5 & 6 & 6 &  & 17 & 0 & 1.94 & 0.29 & 0.65 \\ \hline
  \end{tabular}
  \caption[Receive statistieken gemaakt door Balltime AI voor Lindemans Aalst in set 2]{\label{tab:PL1ReceiveAalst2}Receive statistieken gemaakt door Balltime AI voor Lindemans Aalst in set 2.}
\end{table}

\begin{table}[ht!]
  \centering
  \scriptsize
  \begin{tabular}{|l|c|c|c|c|c|c|c|} \hline
    \textbf{Name} & Set Ast & Set TA & Set SE & A/S & PCT & Dig DS & Dig DE \\ \hline
    Timo Lohmus & 2 & 0 & 1 & 1 & 2 & 0 & 3 \\
    Max Schulz & 2 & 0 & 1 & 0.5 & 1 & 0 & 5 \\
    Hiago Crins &  &  &  &  &  & 1 & 0 \\
    Mihkel Varblane &  &  &  &  &  & 2 & 0 \\
    Alvaro Gimeno Rubio & 0 & 2 & 0 & 0 & 3 & 0 & 3 \\
    Robbe Ponseele &  &  &  &  &  &  &  \\
    Lucas Lorente López & 10 & 18 & 0 & 10 & 0.56 & 1 & 0 \\
    Bert Dufraing & 1 &  &  &  & 2 & 0 & 2 \\
    Lindemans Aalst & 8 & 23 & 0 & 8 & 0.35 & 6 & 1 \\
    Greenyard Maaseik & 12 & 24 & 0 & 12 & 0.52 & 11 & 0 \\ \hline
  \end{tabular}
  \caption[Setting en digging statistieken gemaakt door Balltime AI voor Lindemans Aalst in set 2]{\label{tab:PL1SetDigAalst2}Setting en digging statistieken gemaakt door Balltime AI voor Lindemans Aalst in set 2.}
\end{table}

\begin{table}[ht!]
  \centering
  \scriptsize
  \begin{tabular}{|l|c|c|c|c|c|c|c|c|c|} \hline
    \textbf{Name} & Attack K & E & TA & Atk\% & Kill\% & K/S & Error\% & Block BS & BA \\ \hline
    Timo Lohmus & 1 & 6 & 0.33 & 0.5 & 0.5 & 0.17 & 0 &  &  \\
    Max Schulz & 3 & 9 & 0.22 & 0.5 & 0.56 & 0.33 & 0 & 4 & 0 \\
    Hiago Crins & 0 & 1 & 0 & 0 & 0 & 0 & 0 & 0 & 2 \\
    Mihkel Varblane & 2 & 0 & 2 & 1 & 1 & 0 & 1 & 2 & 0 \\
    Alvaro Gimeno Rubio & 3 & 2 & 6 & 0.17 & 0.25 & 0.5 & 0.33 & 0 & 1 \\
    Robbe Ponseele &  &  &  &  &  &  &  &  &  \\
    Lucas Lorente López &  &  &  &  &  &  &  & 0 & 1 \\
    Bert Dufraing & 0.5 & 0.5 &  &  &  &  &  &  &  \\
    Lindemans Aalst & 10 & 4 & 26 & 0.23 & 0.33 & 0.09 & 0.38 & 10 & 0.15 \\
    Greenyard Maaseik & 13 & 6 & 24 & 0.29 & 0.47 & 0.54 & 0.25 & 10 & 0 \\ \hline
  \end{tabular}
  \caption[Attacking en blocking statistieken gemaakt door Balltime AI voor Lindemans Aalst in set 2]{\label{tab:PL1AttBlockAalst2}Attacking en blocking statistieken gemaakt door Balltime AI voor Lindemans Aalst in set 2.}
\end{table}

\begin{figure}
  \centering
  \includegraphics[width=\textwidth]{PL1_AM/SET2/ROT_STATS.png}
  \caption{\label{fig:PL1_ROT_STATS_2}Rotatie statistieken gemaakt door Balltime AI voor Lindemans Aalst in set 2.}
\end{figure}


\subsection{Vergelijking van de opslagsnelheden}
In tabel \ref{tab:PL1ServeMan2} zijn de manueel gemeten opslagsnelheden weergegeven. In \ref{tab:PL1ServeAI2} is de opslagsnelheden weergegeven gemaakt door Balltime AI. De eerste kolom geeft de setstanden aan. De tweede kolom geeft de speler van Lindemans Aalst aan die serveert. De derde kolom geeft de speler van Greenyard Maaseik aan die serveert en de vierde kolom geeft de snelheid in km/u aan. Op figuur \ref{fig:PL1_Serve} is een voorbeeld van de opslagmeting van Balltime AI te zien. De snelheid is weergegeven in de rechterbovenhoek van het scherm.

\begin{figure}
  \centering
  \includegraphics[width=\textwidth]{PL1_AM/opslag.png}
  \caption{\label{fig:PL1_Serve}Voorbeeld van opslagmeting van Balltime AI.}
\end{figure}

Ook in de tweede set, zie tabel \ref{tab:PL1ServeAI2}, valt op dat Balltime AI niet alle snelheden heeft opgemeten. In deze set zijn er 23 metingen gedaan van de 46 opslagen, 50\% van de opslagen. Dit is een verbetering ten opzichte van de eerste set, maar er zijn nog steeds veel snelheden die niet zijn gemeten. De snelheden die wel zijn gemeten, zijn ook hier weer verschillend van de manueel gemeten snelheden. Bij stand 10-9 werd manueel 80 km/u gemeten, terwijl Balltime AI 114 km/u registreerde. Een kleinere afwijking was dan wel weer te vinden bij stand 16-13, waar Balltime AI een snelheid van 41 km/u aangaf tegenover 45 km/u bij de manuele meting. Balltime AI gaf op sommige momenten wel veel hogere waardes, zoals 122 km/u bij 13-11, waar manueel slechts 55 km/u werd gemeten. In deze set lijken er ook enkele consistentere snelheden te zijn gemeten, zoals bij 1-0, waar manueel de meting 93 km/u aangaf en Balltime AI 94 km/u aangaf.

\begin{table}[ht!]
  \centering
  \scriptsize
  \begin{tabular}{|c|c|c|c|} \hline
    L.A.-G.M. & L.A. & G.M. & km/u \\ \hline
    0-0 &  & 19 & 116 \\
    1-0 & 7 & & 93 \\
    1-1 &  & 14 & 50 \\
    2-1 & 1 & & 51 \\
    3-1 & 1 & & 53 \\
    3-2 &  & 10 & 95 \\
    4-2 & 13 & & 89 \\
    4-3 &  & 15 & 71 \\
    5-3 & 9 & & 97 \\
    5-4 &  & 2 & 90 \\
    6-4 & 11 & & 82 \\
    6-5 &  & 4 & 61 \\
    7-5 & 14 & & 53 \\
    7-6 &  & 19 & 116 \\
    8-6 & 7 &  & 101 \\
    8-7 &  & 14 & 55 \\
    8-8 &  & 14 & 58 \\
    9-8 & 7 & & 60 \\
    9-9 &  & 10 & 106 \\
    10-9 & 13 & & 80 \\
    11-9 & 13 & & 95 \\
    12-9 & 13 & & 97 \\
    12-10 &  & 15 & 109 \\
    12-11 &  & 15 & 92 \\
    13-11 & 9 &  & 55 \\
    13-12 &  & 2 & 93 \\
    14-12 & 11 &  & 98 \\
    14-13 &  & 4 & 53 \\
    15-13 & 14 &  & 51 \\
    16-13 & 14 &  & 45 \\
    16-14 &  & 19 & 114 \\
    17-14 & 7 &  & 105 \\
    18-14 & 7 &  & 100 \\
    18-15 &  & 14 & 50 \\
    18-16 &  & 14 & 58 \\
    19-16 & 3 &  & 106 \\
    20-16 & 3 &  & 98 \\
    20-17 &  & 10 & 105 \\
    21-17 & 13 &  & 97 \\
    22-17 & 13 &  & 106 \\
    23-17 & 13 &  & 100 \\
    23-18 &  & 15 & 58 \\
    23-19 &  & 15 & 79 \\
    24-19 & 9 &  & 98 \\
    24-20 &  & 2 & 108 \\
    24-21 &  & 2 & 101 \\
    25-21 &  &  &  \\ \hline
  \end{tabular}
  \caption[Manueel gemeten opslagsnelheden tijdens set 2]{\label{tab:PL1ServeMan2}Manueel gemeten opslagsnelheden tijdens set 2.}
\end{table}

\begin{table}[ht!]
  \centering
  \scriptsize
  \begin{tabular}{|c|c|c|c|} \hline
    L.A.-G.M. & L.A. & G.M. & km/u \\ \hline
    0-0 &  & 19 & - \\
    1-0 & 7 & & 94 \\
    1-1 &  & 14 & 56 \\
    2-1 & 1 & & 55 \\
    3-1 & 1 & & 59 \\
    3-2 &  & 10 & - \\
    4-2 & 13 & & - \\
    4-3 &  & 15 & - \\
    5-3 & 9 & & - \\
    5-4 &  & 2 & 95 \\
    6-4 & 11 & & - \\
    6-5 &  & 4 & - \\
    7-5 & 14 & & 71 \\
    7-6 &  & 19 & - \\
    8-6 & 7 &  & 70 \\
    8-7 &  & 14 & 54 \\
    8-8 &  & 14 & 43 \\
    9-8 & 7 & & 73 \\
    9-9 &  & 10 & 92 \\
    10-9 & 13 & & 114 \\
    11-9 & 13 & & 109 \\
    12-9 & 13 & & 98 \\
    12-10 &  & 15 & - \\
    12-11 &  & 15 & 74 \\
    13-11 & 9 &  & 122 \\
    13-12 &  & 2 & - \\
    14-12 & 11 &  & - \\
    14-13 &  & 4 & 52 \\
    15-13 & 14 &  & 113 \\
    16-13 & 14 &  & 41 \\
    16-14 &  & 19 & - \\
    17-14 & 7 &  & - \\
    18-14 & 7 &  & 53 \\
    18-15 &  & 14 & 49 \\
    18-16 &  & 14 & - \\
    19-16 & 3 &  & - \\
    20-16 & 3 &  & - \\
    20-17 &  & 10 & - \\
    21-17 & 13 &  & 118 \\
    22-17 & 13 &  & - \\
    23-17 & 13 &  & - \\
    23-18 &  & 15 & - \\
    23-19 &  & 15 & - \\
    24-19 & 9 &  & 33 \\
    24-20 &  & 2 & - \\
    24-21 &  & 2 & - \\
    25-21 &  &  &  \\ \hline
  \end{tabular}
  \caption[Gemeten opslagsnelheden door Balltime AI tijdens set 2]{\label{tab:PL1ServeAI2}Gemeten opslagsnelheden door Balltime AI tijdens set 2.}
\end{table}

\section{Conclusie}
De vergelijking tussen de AI-gegenereerde statistieken en de handmatige invoer onthulde specifieke verschillen in registratie en beoordeling. Bij de opslagstatistieken gebruikt Balltime AI een numeriek scoresysteem en is minder kritisch dan de handmatige invoer met symbolen. Echter, bij deze statistieken kijkt de AI wel naar hoe de tegenstander omgaat met de opslag. Dit vormt geen probleem voor de staf van Lindemans Aalst, aangezien deze gegevens ook nog steeds bruikbaar zijn voor hun analyses. Dit is ook het geval voor de receptiestatistieken.

Bij de spelverdeling, aanvalstatistieken en verdedigingsstatistieken werd de kwaliteit niet beoordeeld door Balltime AI. De blokstatistieken hadden het grootste verschil. De AI gaf geen duidelijke indicatie van wie het blok uitvoerde, terwijl dit bij de handmatige invoer wel het geval was. Dit is een belangrijk aandachtspunt voor de verdere ontwikkeling van het AI-systeem, aangezien deze informatie cruciaal is voor de analyse van de prestaties van individuele spelers.

De opslagsnelheden werden door Balltime AI niet altijd geregistreerd, wat een beperking vormt voor de volledigheid van de gegevens. De snelheden die wel werden gemeten, vertoonden aanzienlijke variaties ten opzichte van de handmatige metingen. Dit wijst op een mogelijke inconsistentie in de nauwkeurigheid van de AI-registratie, maar de manuele invoer was ook niet altijd even consistent. De snelheden worden ook niet elke match manueel gemeten, waardoor hiervoor de AI wel een meerwaarde kan bieden. De meting van de snelheden bij Balltime AI was ook nog in een testfase.

%TODO: zeggen dat het bij training wel voordeel gaat hebben, maar dat het bij wedstrijden niet altijd even nuttig is
