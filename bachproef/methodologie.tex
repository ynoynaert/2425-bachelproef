%%=============================================================================
%% Methodologie
%%=============================================================================

\chapter{\IfLanguageName{dutch}{Methodologie}{Methodology}}%
\label{ch:methodologie}

%% TODO: In dit hoofstuk geef je een korte toelichting over hoe je te werk bent
%% gegaan. Verdeel je onderzoek in grote fasen, en licht in elke fase toe wat
%% de doelstelling was, welke deliverables daar uit gekomen zijn, en welke
%% onderzoeksmethoden je daarbij toegepast hebt. Verantwoord waarom je
%% op deze manier te werk gegaan bent.
%% 
%% Voorbeelden van zulke fasen zijn: literatuurstudie, opstellen van een
%% requirements-analyse, opstellen long-list (bij vergelijkende studie),
%% selectie van geschikte tools (bij vergelijkende studie, "short-list"),
%% opzetten testopstelling/PoC, uitvoeren testen en verzamelen
%% van resultaten, analyse van resultaten, ...
%%
%% !!!!! LET OP !!!!!
%%
%% Het is uitdrukkelijk NIET de bedoeling dat je het grootste deel van de corpus
%% van je bachelorproef in dit hoofstuk verwerkt! Dit hoofdstuk is eerder een
%% kort overzicht van je plan van aanpak.
%%
%% Maak voor elke fase (behalve het literatuuronderzoek) een NIEUW HOOFDSTUK aan
%% en geef het een gepaste titel.


\subsection{Fase 1: Literatuurstudie}
De literatuurstudie is bedoeld om inzicht te krijgen in de huidige stand van zaken met betrekking tot AI-systemen die geschikt zijn voor het verzamelen en analyseren van sportstatistieken. Ook wordt gekeken naar AI-modellen die in recent onderzoek effectief bleken voor sportanalyse. Daarnaast wordt ook de rol van sensors en camera's onderzocht. Waarom zijn statistieken belangrijk in de sportwereld en specifiek in volleybal? Deze vraag wordt beantwoord door het analyseren van academische publicaties en praktijkvoorbeelden. Daarnaast worden whitepapers en technische documentatie van bekende AI-tools, zoals Hudl, DataVolley, Second Spectrum en Balltime AI, bestudeerd.

De focus ligt hierbij op het beoordelen van nauwkeurigheid, snelheid, schaalbaarheid en kosten van deze systemen, omdat deze factoren belangrijk zijn voor de toepassing binnen Lindemans Aalst. Welke technologieën worden momenteel gebruikt voor het automatiseren van sportstatistieken? Dit wordt onderzocht door een overzicht te creëren van bestaande systemen en technologieën.

Aan het einde van deze fase wordt een overzichtsrapport opgesteld met de bevindingen, waarin ook een vergelijkingstabel wordt opgenomen die de sterke en zwakke punten van elk systeem schetst. Dit rapport vormt de basis voor de selectie van systemen die in latere fases verder onderzocht zullen worden.
\subsection{Fase 2: Interviews en requirementsanalyse}
In de tweede fase worden gestructureerde interviews afgenomen met de coaches, data-analisten en technische staf van Lindemans Aalst. Het doel van deze interviews is om een duidelijk beeld te krijgen van de functionele en technische eisen die de club stelt aan een geautomatiseerd statistieken systeem. Daarnaast wordt onderzocht in hoeverre de technische staf en spelers bereid zijn AI-technologie te omarmen en welke barrières er zijn bij implementatie. Ook wordt extra aandacht besteed aan de gebruiksvriendelijkheid van de systemen en de mate van benodigde technische kennis bij gebruikers. "Hoe presteren bestaande AI-systemen voor het verzamelen en analyseren van volleybalstatistieken?" en "Welke technologieën worden momenteel gebruikt voor het automatiseren van sportstatistieken?". Deze vragen worden besproken met stakeholders, waarbij hun ervaringen en verwachtingen worden genoteerd.

De requirementsanalyse richt zich op de statistieken die essentieel zijn voor de club en op specifieke analyses die coaches tijdens wedstrijden en trainingen nodig hebben. De inzichten uit deze gesprekken worden verwerkt in een requirementsdocument, dat de functionele en technische vereisten vastlegt. Dit document vormt een referentiepunt voor de vergelijkende analyse van de AI-systemen in de volgende fase.
\subsection{Fase 3: Vergelijkende studie van AI-systemen}
In deze fase worden de geselecteerde AI-systemen geëvalueerd op basis van nauwkeurigheid, snelheid en gebruiksgemak, om objectief te bepalen welke oplossing het best aansluit bij de eisen van Lindemans Aalst. Hiervoor wordt gebruikgemaakt van eerder verzamelde datasets van match- en trainingsgegevens. "Hoe presteren bestaande AI-systemen voor het verzamelen en analyseren van volleybalstatistieken?" wordt verder onderzocht door de prestaties van de systemen in de praktijk te testen en te vergelijken met handmatig geregistreerde data.

De systemen worden getest op hun vermogen om deze data nauwkeurig en snel te verwerken. Hierbij wordt niet alleen gekeken naar nauwkeurigheid en snelheid, maar ook naar de integratie met de bestaande infrastructuur van Lindemans Aalst. Daarnaast worden skeletgebaseerde AI-systemen zoals beschreven in de literatuurstudie meegenomen in de evaluatie, om hun potentieel binnen volleybalanalyse te beoordelen. De resultaten worden gepresenteerd in een rapport met grafieken en tabellen die de prestaties van elk systeem visueel weergeven. Het rapport bevat ook een concrete aanbeveling van het systeem dat het meest geschikt lijkt voor implementatie, op basis van de vergelijking.
\subsection{Fase 4: Proof of concept en praktijktest}
Na de vergelijkende studie wordt het aanbevolen systeem geïmplementeerd als proof of concept (PoC) bij Lindemans Aalst, op voorwaarde dat het seizoen nog bezig is. Deze praktijktest heeft als doel te valideren of het systeem daadwerkelijk voldoet aan de functionele en technische eisen van de club. Gedurende deze periode verzamelt het systeem automatisch gegevens tijdens trainingen en wedstrijden, zodat coaches en analisten het gebruiksgemak en de nauwkeurigheid van de automatisering kunnen evalueren. De prestaties van AI worden vergeleken met de traditionele methode om te bepalen of AI een significante meerwaarde biedt."

De prestaties van het PoC-systeem worden vergeleken met handmatig geregistreerde gegevens en de feedback van de coaches over de gebruiksvriendelijkheid wordt verzameld. Aan het einde van deze fase wordt een validatierapport opgesteld, waarin de kwantitatieve resultaten van het geautomatiseerde systeem zijn opgenomen, evenals eventuele aanbevelingen voor optimalisatie.
\subsection{Fase 5: Analyse en eindrapport}
In de laatste fase worden de resultaten van het onderzoek geanalyseerd en samengebracht in een eindrapport. Dit rapport bevat een grondige analyse van de prestaties van het PoC-systeem en conclusies over de mate waarin het systeem de doelstellingen van Lindemans Aalst ondersteunt. "Waarom zijn statistieken belangrijk in de sportwereld en specifiek in volleybal?" wordt in deze fase opnieuw geëvalueerd op basis van de praktische resultaten.

De data uit de eerdere fases worden statistisch geanalyseerd om objectieve conclusies te trekken over de invloed van automatisering op de nauwkeurigheid en snelheid van statistiekenregistratie. Het rapport bevat zowel aanbevelingen voor de club over de definitieve implementatie van het AI-systeem als suggesties voor toekomstige optimalisaties.
