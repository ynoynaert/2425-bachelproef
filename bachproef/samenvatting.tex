%%=============================================================================
%% Samenvatting
%%=============================================================================

% De "abstract" of samenvatting is een kernachtige (~ 1 blz. voor een
% thesis) synthese van het document.
%
% Een goede abstract biedt een kernachtig antwoord op volgende vragen:
%
% 1. Waarover gaat de bachelorproef?
% 2. Waarom heb je er over geschreven?
% 3. Hoe heb je het onderzoek uitgevoerd?
% 4. Wat waren de resultaten? Wat blijkt uit je onderzoek?
% 5. Wat betekenen je resultaten? Wat is de relevantie voor het werkveld?
%
% Daarom bestaat een abstract uit volgende componenten:
%
% - inleiding + kaderen thema
% - probleemstelling
% - (centrale) onderzoeksvraag
% - onderzoeksdoelstelling
% - methodologie
% - resultaten (beperk tot de belangrijkste, relevant voor de onderzoeksvraag)
% - conclusies, aanbevelingen, beperkingen
%
% LET OP! Een samenvatting is GEEN voorwoord!

%%---------- Nederlandse samenvatting -----------------------------------------
%
% Als je je bachelorproef in het Engels schrijft, moet je eerst een
% Nederlandse samenvatting invoegen. Haal daarvoor onderstaande code uit
% commentaar.
% Wie zijn bachelorproef in het Nederlands schrijft, kan dit negeren, de inhoud
% wordt niet in het document ingevoegd.

\IfLanguageName{english}{%
\selectlanguage{dutch}
\chapter*{Samenvatting}
\lipsum[1-4]
\selectlanguage{english}
}{}

%%---------- Samenvatting -----------------------------------------------------
% De samenvatting in de hoofdtaal van het document

\chapter*{\IfLanguageName{dutch}{Samenvatting}{Abstract}}
In de sportwereld, en specifiek in volleybal, spelen wedstrijd- en spelerstatistieken een cruciale rol bij prestatieverbetering en strategische besluitvorming. Bij volleybalclub Lindemans Aalst worden wedstrijdstatistieken momenteel handmatig bijgehouden, terwijl er tijdens trainingen zelfs geen gegevens worden vastgelegd. Dit onderzoek richt zich op de mogelijkheden en voordelen van het automatiseren van deze statistieken met behulp van AI-technologie, om zo een competitief voordeel te behalen.

De centrale onderzoeksvraag luidt: \textit{"Welke bestaande AI-oplossing voor het automatiseren van volleybalstatistieken biedt de meeste voordelen voor Lindemans Aalst, zowel tijdens wedstrijden als trainingen?"} Het doel is om via een vergelijkende studie de meest geschikte AI-technologie te selecteren en implementeren, rekening houdend met nauwkeurigheid, gebruiksgemak, kosten en toepasbaarheid binnen de clubcontext.

De methodologie bestaat uit een literatuurstudie naar bestaande AI-oplossingen voor sportanalyse, interviews met stakeholders binnen de club om functionele en technische vereisten in kaart te brengen en een vergelijkende studie van AI-systemen. De best presterende oplossing wordt getest in een proof-of-concept binnen de dagelijkse werking van Lindemans Aalst.

Uit de analyse van de praktijktesten bleek dat de AI-oplossing in staat was de meeste statistieken even accuraat kan registeren als de handmatige methode. Op basis van de resultaten van het onderzoek kan geconcludeerd worden dat automatisering van statistieken via AI een meerwaaarde biedt voor Lindemans Aalst. Het stelt de technische staf in staat om ook tijdens trainingen waardevolle data te verzamelen. 
