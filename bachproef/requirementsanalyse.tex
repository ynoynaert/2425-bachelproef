\section{Requirementsanalyse}
\label{ch:requirementsanalyse}

\subsection{Inleiding}
Dit deel bevat de functionele en technische vereisten voor een geautomatiseerd statistisch analysesysteem voor Lindemans Aalst, gebaseerd op gesprekken met de technische staf en het bestuur van de club. Het doel van het systeem is om prestatiegegevens van spelers en teams te verzamelen, analyseren en presenteren om strategische beslissingen te ondersteunen tijdens trainingen en wedstrijden.

\subsection{Doelen}
\begin{itemize}
  \item Automatiseren van het verzamelen van data en het uitvoeren van analyses.
  \item Verbeteren van de efficiëntie en nauwkeurigheid van de scoutingprocessen.
  \item Real-time statistieken en inzichten leveren tijdens wedstrijden.
  \item Bieden van gedetailleerde en op maat gemaakte statistieken voor coaches, assistent-coaches, en andere betrokkenen.
\end{itemize}

\subsection{Functionele eisen}
\subsubsection{Vereiste statistieken en data}
\begin{itemize}
  \item Scoringspercentages in side-out en transitie, inclusief analyses per rotatie en situatie.
  \item Foutenanalyse, waaronder het aantal fouten per speler en de rotaties waarin deze fouten voorkomen.
  \item Setteranalyse, waarbij de keuzes van de setter tijdens verschillende recepties en rotaties worden geëvalueerd.
  \item Spelerprestaties, zowel individueel als in teamverband, inclusief serveer- en ontvangstpatronen.
  \item Tegenstander Analyse, met focus op specifieke zwakke punten van de tegenstander, inclusief individuele analyses van de tegenstander (zoals receptie en serveervariaties).
\end{itemize}

\subsubsection{Automatisering van processen}
\begin{itemize}
  \item Automatisering van gegevensinvoer en -verwerking van scoutingdata.
  \item Automatisering van analyses zoals serveervariaties, aanvalsrichtingen, en defensieve patronen.
  \item Het systeem moet in staat zijn om gegevens zoals serveer- en ontvangstinformatie automatisch te verzamelen en te verwerken.
\end{itemize}

\subsubsection{Outputformaten}
\begin{itemize}
  \item Dashboards voor real-time feedback tijdens wedstrijden en trainingen.
  \item Rapporten na wedstrijden en trainingen met gedetailleerde analyses van team- en individuele prestaties.
  \item Live feedback die coaches in staat stelt om direct aanpassingen te maken op basis van de verzamelde gegevens.
  \item Flexibiliteit om verschillende soorten output te genereren, zoals grafieken, tabellen en visuele dataweergaves.
\end{itemize}

\subsubsection{Real-time verwerking}
\begin{itemize}
  \item Het systeem moet real-time data verwerken tijdens wedstrijden, idealiter binnen een set.
  \item De updates van statistieken moeten snel en accuraat zijn, zodat coaches en spelers direct toegang hebben tot de meest actuele gegevens.
\end{itemize}

\subsubsection{Gebruikersinterface}
\begin{itemize}
  \item Het systeem moet gebruiksvriendelijk zijn, met een interface die gemakkelijk te navigeren is voor zowel technisch als niet-technisch personeel.
  \item Visuele weergave van data moet duidelijk en overzichtelijk zijn, met mogelijk gebruik van kleurcoderingen om snel belangrijke informatie te identificeren.
  \item Het systeem moet beschikbaar zijn op verschillende apparaten, inclusief mobiele apparaten (smartphones en tablets) voor flexibele toegang.
\end{itemize}

\subsection{Technische eisen}
\subsubsection{Integratie}
\begin{itemize}
  \item Het systeem moet kunnen worden geïntegreerd met bestaande software zoals DataVolley en VolleyMetrics voor een naadloze samenwerking en gegevensuitwisseling.
  \item Het systeem moet multi-platform compatibel zijn, zodat het op verschillende besturingssystemen en apparaten werkt, inclusief Windows, macOS, iOS en Android.
\end{itemize}

\subsubsection{Betrouwbaarheid}
\begin{itemize}
  \item Het systeem moet zeer betrouwbaar zijn, met minimale kans op technische storingen tijdens wedstrijden of trainingen.
  \item Er moet een back-upsysteem en technische ondersteuning beschikbaar zijn om te zorgen dat het systeem goed functioneert, zelfs in het geval van een storing of andere technische problemen.
\end{itemize}

\subsubsection{Gebruik van AI}
\begin{itemize}
  \item Het systeem moet gebruik kunnen maken van AI voor het automatisch analyseren van spelpatronen, zoals de richtingen van aanvallers en de keuzes van de setter.
  \item Machine learning kan worden ingezet om het systeem in staat te stellen zich aan te passen op basis van historische gegevens en toekomstige voorspellingen.
\end{itemize}

\subsection{Implementatievereisten}
\subsubsection{Training en ondersteuning}
\begin{itemize}
  \item Er moeten trainingssessies worden aangeboden voor zowel technische staf als vrijwilligers, zodat zij het systeem effectief kunnen gebruiken.
  \item Een technische expert moet beschikbaar zijn om ondersteuning te bieden, vooral tijdens de eerste implementatieperiode en wedstrijden.
\end{itemize}

\subsubsection{Konsten en return on investment}
\begin{itemize}
  \item Het systeem moet kosten-efficiënt zijn, met een duidelijk kosten-batenanalyse. De investering moet in verhouding staan tot de voordelen die het systeem biedt in termen van verbeterde prestaties en efficiëntie.
\end{itemize}
